%\input{tcilatex}
%\input{tcilatex}
%\input{tcilatex}
%\input{tcilatex}


\documentclass[12pt,a4paper]{article}
%%%%%%%%%%%%%%%%%%%%%%%%%%%%%%%%%%%%%%%%%%%%%%%%%%%%%%%%%%%%%%%%%%%%%%%%%%%%%%%%%%%%%%%%%%%%%%%%%%%%%%%%%%%%%%%%%%%%%%%%%%%%%%%%%%%%%%%%%%%%%%%%%%%%%%%%%%%%%%%%%%%%%%%%%%%%%%%%%%%%%%%%%%%%%%%%%%%%%%%%%%%%%%%%%%%%%%%%%%%%%%%%%%%%%%%%%%%%%%%%%%%%%%%%%%%%
\usepackage{amsmath}
\usepackage{amsfonts}
\usepackage{amssymb}
\usepackage{fancyhdr}
\usepackage{hyperref}
\usepackage{enumitem}

\setcounter{MaxMatrixCols}{10}
%TCIDATA{OutputFilter=Latex.dll}
%TCIDATA{Version=5.50.0.2953}
%TCIDATA{<META NAME="SaveForMode" CONTENT="1">}
%TCIDATA{BibliographyScheme=Manual}
%TCIDATA{LastRevised=Sunday, April 27, 2025 15:50:30}
%TCIDATA{<META NAME="GraphicsSave" CONTENT="32">}

\newtheorem{theorem}{Theorem}
\newtheorem{acknowledgement}[theorem]{Acknowledgement}
\newtheorem{algorithm}[theorem]{Algorithm}
\newtheorem{axiom}[theorem]{Axiom}
\newtheorem{case}[theorem]{Case}
\newtheorem{claim}[theorem]{Claim}
\newtheorem{conclusion}[theorem]{Conclusion}
\newtheorem{condition}[theorem]{Condition}
\newtheorem{conjecture}[theorem]{Conjecture}
\newtheorem{corollary}[theorem]{Corollary}
\newtheorem{criterion}[theorem]{Criterion}
\newtheorem{definition}[theorem]{Definition}
\newtheorem{example}[theorem]{Example}
\newtheorem{exercise}[theorem]{Exercise}
\newtheorem{lemma}[theorem]{Lemma}
\newtheorem{notation}[theorem]{Notation}
\newtheorem{problem}[theorem]{Problem}
\newtheorem{proposition}[theorem]{Proposition}
\newtheorem{remark}[theorem]{Remark}
\newtheorem{solution}[theorem]{Solution}
\newtheorem{summary}[theorem]{Summary}
\newenvironment{proof}[1][Proof]{\noindent\textbf{#1.} }{\ \rule{0.5em}{0.5em}}
\renewcommand{\headrulewidth}{0.4pt}
\renewcommand{\footrulewidth}{0.4pt}

\input{tcilatex}

\begin{document}

\title{Mahler and Yum (2024)}
\author{Alessandro Di Nola, Robert Kirkby and Haomin Wang}
\date{}
\maketitle

\section{Model description}

Brief description of the replication of Mahler and Yum (2024). We follow
closely their notation. The only departure in terms of model assumptions: we
eliminate lagged health effort. We do this change since the code runs much
faster. Moreover, it may not be such an important feature.

\begin{itemize}
\item State variables:

\begin{itemize}
\item Asset holdings $a$

\item Health status $h$

\item Labor productivity $z$

\item Permanent types:

\begin{itemize}
\item Fixed education $e\in \left\{ 0,1\right\} $ (non-college, college)

\item Patience $\beta \in \left\{ 0,1\right\} $ (low, high)

\item Ability $\theta \in \left\{ 0,1\right\} $ (low, high)

\item Fixed health $\eta \in \left\{ 0,1\right\} $ (bad, good)
\end{itemize}
\end{itemize}

\item Choice variables:

\begin{itemize}
\item Labor supply $n\in \left\{ 0,n_{PT},n_{FT}\right\} $

\item Health effort $f\in \lbrack 0,1]$

\item Next-period assets $a^{\prime }$
\end{itemize}

\item Individual problem%
\begin{equation*}
V_{j}\left( a,h,z,\overline{e},\overline{\beta },\overline{\theta },%
\overline{\eta }\right) =\max_{c,a^{\prime },n,f}\left\{ u\left( c,n,f,h,%
\overline{e}\right) +\overline{\beta }s_{j}(h,\overline{e})E\left[
V_{j+1}(a^{\prime },h^{\prime },z^{\prime },\overline{e},\overline{\beta },%
\overline{\theta },\overline{\eta })\right] \right\} 
\end{equation*}%
subject to%
\begin{equation*}
c+a^{\prime }=(1+r)a+T(c,h,n)+w_{j}\left( h,z,\overline{e},\overline{\theta }%
\right) n-\mathcal{T}\left( w_{j}\left( h,z,\overline{e},\overline{\theta }%
\right) n,y_{bar}\right) \text{, if }j<J_{R}
\end{equation*}%
and%
\begin{equation*}
c+a^{\prime }=(1+r)a+P(\overline{e})\text{, if }j\geq J_{R}
\end{equation*}%
where $T\left( \cdot \right) $ denotes transfers and $\mathcal{T}\left(
\cdot \right) $ denotes income taxes. The expected value function is defined
as%
\begin{equation*}
E\left[ V_{j+1}(a^{\prime },h^{\prime },z^{\prime },\overline{e},\overline{%
\beta },\overline{\theta },\overline{\eta })\right] =\sum_{z^{\prime
}}\sum_{h^{\prime }}V_{j+1}(a^{\prime },h^{\prime },z^{\prime },\overline{e},%
\overline{\beta },\overline{\theta },\overline{\eta })\Pr \left( z^{\prime
}|z\right) \Pr \left( h^{\prime }|h,f,\overline{e},\overline{\eta },j\right) 
\end{equation*}%
Note that the transition probability for health $h$ depends on health effort 
$f$, a decision variable. Therefore $h$ is a semi-exogenous state. Moreover,
the transition probability for health is also affected by fixed education
and health types, and by age.
\end{itemize}

\subsection{Functional Forms}

\begin{itemize}
\item Transition probability for health status $h\in \left\{ 0,1\right\} $,
where $h=0$ is \textquotedblleft unhealth\textquotedblright\ and $h=1$ is
\textquotedblleft healthy\textquotedblright . The probability of bein
healthy in the next period is an increasing function of effort $f\in \lbrack
0,1]$ modelled as a logistic curve:%
\begin{equation*}
\Pr \left( h^{\prime }=1|h,f,\overline{e},\overline{\eta },j\right) =\frac{1%
}{1+\exp \left( -\left( \pi _{j}^{0}+\lambda _{1}f+\delta h+\gamma _{1}%
\overline{e}+\gamma _{2}\overline{\eta }\right) \right) }
\end{equation*}%
Note that $\pi _{j}^{0}$ is a shifter that depends on age\footnote{%
The original formulation in the paper has $\pi _{j}^{0}+\gamma _{3}A_{i}$
where $\pi _{j}^{0}$ is the constant (baseline age group) and $A_{i}$ is a
vector of dummies 0-1 for each each group.}, $f$ is health effort, $h$ is
health in the current period, $\overline{e}$ is fixed education type (0 if
non-college, 1 if college) and $\overline{\eta }$ is fixed health type (0 if
low, 1 if high). Plot $f(x)=\frac{1}{1+\exp (-x)}$: it is defined over $x\in
\left( -\infty ,\infty \right) $, increasing, convex for $x<0$, concave for $%
x>0$, inflexion point at $x=0$ where $f(0)=0.5$. As $x\rightarrow \infty $, $%
f(x)\rightarrow 1$).

\item Tax function $\mathcal{T}\left( y,y_{bar}\right) $ is defined as%
\begin{equation*}
\mathcal{T}\left( y,y_{bar}\right) =y-(1-\tau _{s})y^{1-\tau
_{p}}(y_{bar})^{\tau _{p}}
\end{equation*}%
where $y$ is taxable income and includes only labor earnings $w_{j}\left(
h,z,\overline{e},\overline{\theta }\right) n$.

\item Social transfers are defined as $T(c,h,n)=T_{1}+T_{2}$ where%
\begin{equation*}
T_{1}=\left\{ 
\begin{array}{ll}
\widetilde{c}-c & \text{if }c<\widetilde{c} \\ 
0 & \text{otherwise}%
\end{array}%
\right. 
\end{equation*}%
and%
\begin{equation*}
T_{2}=\left\{ 
\begin{array}{ll}
\widetilde{T}>0 & \text{if }h=0\text{ and }n=0 \\ 
0 & \text{otherwise}%
\end{array}%
\right. 
\end{equation*}%
Note: the transfer $\widetilde{T}>0$ should be paid even if $a^{\prime }>0$?

\item The wage function (i.e. the wage per unit of labor supplied) is
defined as%
\begin{equation*}
w_{j}\left( h,z,\overline{e},\overline{\theta }\right) =\exp \left( \lambda
_{j}\left( h,\overline{e}\right) +\overline{\theta }+z\right) 
\end{equation*}%
where $\lambda _{j}\left( h,\overline{e}\right) $ is the age-dependent
component, $\overline{\theta }$ is a fixed effect (unobserved ability) and $z
$ is an AR(1) productivity shock. The only non-standard assumption is that $%
\lambda _{j}\left( h,\overline{e}\right) $ depends also on health: unhealthy
individuals (with $h=0$) suffer a wage penalty $w_{p}^{\overline{e}}$ which
depends on education:%
\begin{equation*}
\lambda _{j}\left( h,\overline{e}\right) =\zeta _{0}^{\overline{e}}\exp
\left( \zeta _{1}^{\overline{e}}\left( j-1\right) +\zeta _{2}^{\overline{e}%
}\left( j-1\right) ^{2}\right) \left( 1-w_{p}^{\overline{e}}I_{\left\{
h=0\right\} }\right) 
\end{equation*}

\item Pension benefits. Pension benefits depends only on education. It is
not exactly clear how they compute $P(\overline{e})$ from the paper. They
write \textit{\textquotedblleft We initially set these as equal to the
earnings agents would have earned in the period prior to retirement if they
had worked full-time with a median productivity shock value. We then scale
them by a constant\textquotedblright }. In the code we assume that 
\begin{equation*}
P(\overline{e})=\omega \left[ \Pr \left( \overline{\theta }=\overline{\theta 
}_{low}\right) \exp \left( \lambda _{j}\left( h=1,\overline{e}\right) +%
\overline{\theta }_{low}+z_{med}\right) +\Pr \left( \overline{\theta }=%
\overline{\theta }_{high}\right) \exp \left( \lambda _{j}\left( h=1,%
\overline{e}\right) +\overline{\theta }_{high}+z_{med}\right) \right] 
\end{equation*}

\item Preferences. The per-period utility function takes the form:%
\begin{equation*}
u(c,n,f,h,\overline{e})=\kappa \left( h\right) \left( \frac{c^{1-\sigma }}{%
1-\sigma }+b\right) -\phi \left( n;h,\overline{e}\right) -\varphi \left( f;h,%
\overline{e}\right) 
\end{equation*}%
where

\begin{itemize}
\item Utility of consumption is%
\begin{equation*}
\kappa \left( h\right) \left( \frac{c^{1-\sigma }}{1-\sigma }+b\right) 
\end{equation*}%
where $\kappa \left( h\right) =1$ if $h=1$ (healthy) but $\kappa \left(
h\right) =\widetilde{\kappa }<1$.

\item Disutility of labor is%
\begin{equation*}
\phi \left( n;h,\overline{e}\right) =v_{j}\left( h\right) \exp \left(
v_{e}I_{\left\{ \overline{e}=0\right\} }\right) \frac{n^{1+1/\gamma }}{%
1+1/\gamma }
\end{equation*}%
where $v_{e}$ is an extra disutility of working incurred by individual with
non-college education (i.e. $\overline{e}=0$). $v_{j}\left( h\right) $ is an
age-dependent shifter that depends also on health status (to match
employment rates by age and by health).

\item Disutility of health effort is%
\begin{equation*}
\varphi \left( f;h,\overline{e}\right) =\iota _{j}\left( h,\overline{e}%
\right) \frac{f^{1+1/\psi }}{1+1/\psi }
\end{equation*}%
where $\iota _{j}\left( h,\overline{e}=0\right) $ is an age-specific shifter
that depends also on health state and on education fixed type (to match mean
health effort by age, health status and education).
\end{itemize}
\end{itemize}

\section{VFI Toolkit}

\begin{itemize}
\item In the toolkit, the survival rate $s_{j}(h,e)$ cannot depend on state $%
h$.%
\begin{equation*}
V_{j}=u(c)+\beta s_{j}V_{j+1}+\beta \left( 1-s_{j}\right) 0
\end{equation*}%
\begin{equation*}
V_{j}=u(c)+\beta EV_{j+1}\left( h^{\prime }\right) 
\end{equation*}%
Include a dead state into health state%
\begin{equation*}
h=0,1,2
\end{equation*}%
where $h=2$ means dead. Then the transition for $h$ is a 3*3 matrix%
\begin{equation*}
\begin{array}{lccc}
& h=0 & h=1 & h=2 \\ 
h=0\text{ (unhealthy)} & s_{j}\left( 0,e\right) \Pr \left( h^{\prime }=0|0,f,%
\overline{e},\overline{\eta },j\right)  & s_{j}\left( 0,e\right) \Pr \left(
h^{\prime }=1|0,f,\overline{e},\overline{\eta },j\right)  & 1-s_{j}\left(
0,e\right)  \\ 
h=1\text{ (healthy)} & s_{j}(1,e)\Pr \left( h^{\prime }=0|1,f,\overline{e},%
\overline{\eta },j\right)  & s_{j}(1,e)\Pr \left( h^{\prime }=1|1,f,%
\overline{e},\overline{\eta },j\right)  & 1-s_{j}(1,e) \\ 
h=2\text{ (dead)} & 0 & 0 & 1%
\end{array}%
\end{equation*}
\end{itemize}

\end{document}
